\
\documentclass{article}
\usepackage{listings}
\usepackage{hyperref}
\end{itemize}

\title{Quick Start Guide}
\begin{document}
\end{itemize}

\maketitle
\section{Quick Start Guide}
This is a quick guide to help you install and use Yukari Code 2.
\end{itemize}

\sectionsection{Set the Working Directory}
\end{itemize}

Create a directory (folder) at any location for team optimization. In this guide, we'll assume you create a working directory named "Yukari" on the Desktop.
\end{itemize}

The path to the Yukari directory will be as follows:
\end{itemize}

\begin{itemize}
\item macOS: \texttt{/Users/(username)/Desktop/Yukari}
\begin{itemize}
\item Windows: \texttt{C:/Users/(username)/Desktop/Yukari}
\end{itemize}

\sectionsection{Download Codes, Examples, and Documents}
\end{itemize}

Download and unzip all the files from the [Yukari_Code_2 repository](https://github.com/tohru-murakami/Yukari_Code_2).
\end{itemize}

\sectionsection{Put executable code}
\end{itemize}

Copy the executable file \texttt{grouping_ils} (for macOS) or \texttt{grouping_ils.exe} (for Windows) to the top level of the Yukari directory.
\end{itemize}

Also, copy the sample data file \texttt{data16.txt} to the Yukari directory.
\end{itemize}

\sectionsection{Run a Test}
\end{itemize}

Open a terminal application:
\end{itemize}

\begin{itemize}
\item macOS: Terminal.app
\begin{itemize}
\item Windows: Command Prompt
\end{itemize}

Navigate to the Yukari directory:
\end{itemize}

``\texttt{
% cd (path to Yukari directory)
}`\texttt{
\end{itemize}

Execute Yukari Code.
\end{itemize}

On macOS:
\end{itemize}

}`\texttt{
% ./grouping_ils data16.txt
}`\texttt{
\end{itemize}

On Windows:
\end{itemize}

}`\texttt{
> grouping_ils.exe data16.txt
}`\texttt{
\end{itemize}

After about 5 minutes, you will get a result similar to this:
\end{itemize}

}`\texttt{
% ./grouping_ils data16.txt 
data16.txt
0,4,0g,5gH,10,13,6 6 7 4 7 9 ,37.0,39,2,2,1,
1,4,6H,8,11,15,4 6 9 6 4 9 ,34.0,38,0,4,1,
2,4,1g,4g,7H,9,6 6 9 5 10 9 ,45.0,45,2,2,1,
3,4,2g,3g,12,14,8 6 5 10 6 9 ,44.0,44,2,2,0,
Total Value = 160.0
Minimum Compatibility = 4
Total Compatibility = 166
}`\texttt{
\end{itemize}

\sectionsection{Compile the Source Code (macOS & Linux Users)}
\end{itemize}

For macOS and Linux users, you can compile the source code using }make\texttt{. Follow these steps:
\end{itemize}

\sectionsectionsection{Install the Build Tools}
\end{itemize}

Before proceeding, make sure you have the necessary build tools installed. On most systems, this will include a C or C++ compiler (such as }gcc\texttt{ or }clang\texttt{) and }make\texttt{.
\end{itemize}

Alternatively, simply run }make\texttt{, and the operating system will prompt you to install any necessary tools. You can then follow the instructions provided.
\end{itemize}

\sectionsectionsectionsection{macOS}
\end{itemize}

  Install the Xcode command-line tools by running the following command in Terminal:
\end{itemize}

  }`\texttt{
  % xcode-select --install
  }`\texttt{
\end{itemize}

\sectionsectionsectionsection{Linux}
\end{itemize}

Ensure that }gcc\texttt{ and }make\texttt{ are installed. You can do this with the following commands:
\end{itemize}

  }`\texttt{
  % sudo apt update
  % sudo apt install build-essential
  }`\texttt{
\end{itemize}

\sectionsectionsection{Navigate to the Source Code Directory}
\end{itemize}

Go to the directory where the source code files are located. For example, if the source code is in the }Yukari\texttt{ directory, run:
\end{itemize}

}`\texttt{
% cd (path to Yukari directory)/Source\ code
}`\texttt{
\end{itemize}

\sectionsectionsection{Run the }make\texttt{ Command}
\end{itemize}

In the terminal, run the }make\texttt{ command to compile the source code. The }Makefile\texttt{ in the directory should automate the compilation process:
\end{itemize}

}`\texttt{
% make
}`\texttt{
\end{itemize}

If everything is set up correctly, this will generate an executable file called }grouping_ils\texttt{ in the same directory.
\end{itemize}

\sectionsectionsection{Confirm the Compilation}
\end{itemize}

After the }make\texttt{ process finishes, you can check that the executable was created by listing the contents of the directory:
\end{itemize}

}`\texttt{
% ls
}`\texttt{
\end{itemize}

You should see the }grouping_ils\texttt{ executable file.
\end{itemize}

\sectionsectionsection{Run the Program}
\end{itemize}

You can now run the compiled executable as described in the test flight section:
\end{itemize}

}`\texttt{
% ./grouping_ils data16.txt
}``
\end{itemize}

\sectionsection{Utilizing Yukari Code for the Optimization of Actual Anatomy Teams}
\end{itemize}

In order to use the Yukari Code to optimize the assignment of actual anatomy teams, the necessary data must first be prepared.
\end{itemize}

\sectionsectionsection{Administering a Google Form Survey}
\end{itemize}

To gather input from the students enrolled in the anatomy course, create a Google Form with access authentication. The form should include the following questions, each of which will be answered on a 1–5 Likert scale. See the sample survey for details : /GitHub/Yukari_code_2/Yukari Data Manager/Yukari_Preference_Survey.md
\end{itemize}

\sectionsectionsection{Data Collection}
\end{itemize}

Once the Google Form survey has been administered to the students, download the data in Excel format. Check and clean the data. For any students who did not provide responses, Leave their entries blank.
\end{itemize}

The cleaned data will be input into the Yukari Data Manager spreadsheet to generate the data required for use with the Yukari Code.
\n\end{document}